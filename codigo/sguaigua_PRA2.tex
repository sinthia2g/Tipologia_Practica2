% Options for packages loaded elsewhere
\PassOptionsToPackage{unicode}{hyperref}
\PassOptionsToPackage{hyphens}{url}
%
\documentclass[
]{article}
\usepackage{amsmath,amssymb}
\usepackage{lmodern}
\usepackage{iftex}
\ifPDFTeX
  \usepackage[T1]{fontenc}
  \usepackage[utf8]{inputenc}
  \usepackage{textcomp} % provide euro and other symbols
\else % if luatex or xetex
  \usepackage{unicode-math}
  \defaultfontfeatures{Scale=MatchLowercase}
  \defaultfontfeatures[\rmfamily]{Ligatures=TeX,Scale=1}
\fi
% Use upquote if available, for straight quotes in verbatim environments
\IfFileExists{upquote.sty}{\usepackage{upquote}}{}
\IfFileExists{microtype.sty}{% use microtype if available
  \usepackage[]{microtype}
  \UseMicrotypeSet[protrusion]{basicmath} % disable protrusion for tt fonts
}{}
\makeatletter
\@ifundefined{KOMAClassName}{% if non-KOMA class
  \IfFileExists{parskip.sty}{%
    \usepackage{parskip}
  }{% else
    \setlength{\parindent}{0pt}
    \setlength{\parskip}{6pt plus 2pt minus 1pt}}
}{% if KOMA class
  \KOMAoptions{parskip=half}}
\makeatother
\usepackage{xcolor}
\IfFileExists{xurl.sty}{\usepackage{xurl}}{} % add URL line breaks if available
\IfFileExists{bookmark.sty}{\usepackage{bookmark}}{\usepackage{hyperref}}
\hypersetup{
  pdftitle={PRACTICA 2: LIMPIEZA Y VALIDACIÓN DE LOS DATOS},
  pdfauthor={Sinthia Elizabeth Guaigua Guanopatin},
  hidelinks,
  pdfcreator={LaTeX via pandoc}}
\urlstyle{same} % disable monospaced font for URLs
\usepackage[margin=1in]{geometry}
\usepackage{color}
\usepackage{fancyvrb}
\newcommand{\VerbBar}{|}
\newcommand{\VERB}{\Verb[commandchars=\\\{\}]}
\DefineVerbatimEnvironment{Highlighting}{Verbatim}{commandchars=\\\{\}}
% Add ',fontsize=\small' for more characters per line
\usepackage{framed}
\definecolor{shadecolor}{RGB}{248,248,248}
\newenvironment{Shaded}{\begin{snugshade}}{\end{snugshade}}
\newcommand{\AlertTok}[1]{\textcolor[rgb]{0.94,0.16,0.16}{#1}}
\newcommand{\AnnotationTok}[1]{\textcolor[rgb]{0.56,0.35,0.01}{\textbf{\textit{#1}}}}
\newcommand{\AttributeTok}[1]{\textcolor[rgb]{0.77,0.63,0.00}{#1}}
\newcommand{\BaseNTok}[1]{\textcolor[rgb]{0.00,0.00,0.81}{#1}}
\newcommand{\BuiltInTok}[1]{#1}
\newcommand{\CharTok}[1]{\textcolor[rgb]{0.31,0.60,0.02}{#1}}
\newcommand{\CommentTok}[1]{\textcolor[rgb]{0.56,0.35,0.01}{\textit{#1}}}
\newcommand{\CommentVarTok}[1]{\textcolor[rgb]{0.56,0.35,0.01}{\textbf{\textit{#1}}}}
\newcommand{\ConstantTok}[1]{\textcolor[rgb]{0.00,0.00,0.00}{#1}}
\newcommand{\ControlFlowTok}[1]{\textcolor[rgb]{0.13,0.29,0.53}{\textbf{#1}}}
\newcommand{\DataTypeTok}[1]{\textcolor[rgb]{0.13,0.29,0.53}{#1}}
\newcommand{\DecValTok}[1]{\textcolor[rgb]{0.00,0.00,0.81}{#1}}
\newcommand{\DocumentationTok}[1]{\textcolor[rgb]{0.56,0.35,0.01}{\textbf{\textit{#1}}}}
\newcommand{\ErrorTok}[1]{\textcolor[rgb]{0.64,0.00,0.00}{\textbf{#1}}}
\newcommand{\ExtensionTok}[1]{#1}
\newcommand{\FloatTok}[1]{\textcolor[rgb]{0.00,0.00,0.81}{#1}}
\newcommand{\FunctionTok}[1]{\textcolor[rgb]{0.00,0.00,0.00}{#1}}
\newcommand{\ImportTok}[1]{#1}
\newcommand{\InformationTok}[1]{\textcolor[rgb]{0.56,0.35,0.01}{\textbf{\textit{#1}}}}
\newcommand{\KeywordTok}[1]{\textcolor[rgb]{0.13,0.29,0.53}{\textbf{#1}}}
\newcommand{\NormalTok}[1]{#1}
\newcommand{\OperatorTok}[1]{\textcolor[rgb]{0.81,0.36,0.00}{\textbf{#1}}}
\newcommand{\OtherTok}[1]{\textcolor[rgb]{0.56,0.35,0.01}{#1}}
\newcommand{\PreprocessorTok}[1]{\textcolor[rgb]{0.56,0.35,0.01}{\textit{#1}}}
\newcommand{\RegionMarkerTok}[1]{#1}
\newcommand{\SpecialCharTok}[1]{\textcolor[rgb]{0.00,0.00,0.00}{#1}}
\newcommand{\SpecialStringTok}[1]{\textcolor[rgb]{0.31,0.60,0.02}{#1}}
\newcommand{\StringTok}[1]{\textcolor[rgb]{0.31,0.60,0.02}{#1}}
\newcommand{\VariableTok}[1]{\textcolor[rgb]{0.00,0.00,0.00}{#1}}
\newcommand{\VerbatimStringTok}[1]{\textcolor[rgb]{0.31,0.60,0.02}{#1}}
\newcommand{\WarningTok}[1]{\textcolor[rgb]{0.56,0.35,0.01}{\textbf{\textit{#1}}}}
\usepackage{graphicx}
\makeatletter
\def\maxwidth{\ifdim\Gin@nat@width>\linewidth\linewidth\else\Gin@nat@width\fi}
\def\maxheight{\ifdim\Gin@nat@height>\textheight\textheight\else\Gin@nat@height\fi}
\makeatother
% Scale images if necessary, so that they will not overflow the page
% margins by default, and it is still possible to overwrite the defaults
% using explicit options in \includegraphics[width, height, ...]{}
\setkeys{Gin}{width=\maxwidth,height=\maxheight,keepaspectratio}
% Set default figure placement to htbp
\makeatletter
\def\fps@figure{htbp}
\makeatother
\setlength{\emergencystretch}{3em} % prevent overfull lines
\providecommand{\tightlist}{%
  \setlength{\itemsep}{0pt}\setlength{\parskip}{0pt}}
\setcounter{secnumdepth}{-\maxdimen} % remove section numbering
\ifLuaTeX
  \usepackage{selnolig}  % disable illegal ligatures
\fi

\title{PRACTICA 2: LIMPIEZA Y VALIDACIÓN DE LOS DATOS}
\author{Sinthia Elizabeth Guaigua Guanopatin}
\date{2022-06-03}

\begin{document}
\maketitle

{
\setcounter{tocdepth}{2}
\tableofcontents
}
\begin{center}\rule{0.5\linewidth}{0.5pt}\end{center}

\hypertarget{desarrollo}{%
\section{Desarrollo}\label{desarrollo}}

\begin{center}\rule{0.5\linewidth}{0.5pt}\end{center}

\hypertarget{descripciuxf3n-del-dataset.}{%
\subsection{Descripción del
dataset.}\label{descripciuxf3n-del-dataset.}}

\hypertarget{dataset}{%
\subsubsection{Dataset}\label{dataset}}

El conjunto de datos objeto de análisis se ha obtenido a partir de este
enlace en Kaggle, del siguiente link:
\url{https://www.kaggle.com/datasets/syuzai/perth-house-prices}

El fichero tiene formato csv, contiene información de viviendas de
venta.

Los campos de este conjunto de datos, encontramos los siguientes:

address : Dirección física de la propiedad

suburb :localidad específica en Perth;

price : Precio al que se vendió una propiedad (AUD)

bedrooms : Número de dormitorios

bathrooms : Número de baños

garage :Número de plazas de garaje

land\_area : Superficie total del terreno (m\^{}2)

floor\_area : Superficie interior (m\^{}2)

buil\_year : Año en que se construyó la propiedad

CBD\_dist : Distancia desde el centro de Perth (m)

nearest\_stn : La estación de transporte público más cercana a la
propiedad

nearest\_stn\_dst : La distancia de la estación más cercana (m)

date\_sold : Mes y año en que se vendió la propiedad

NEAREST\_SCH\_RANK: Ranking de cercanía a la escuela

\textbf{¿Por qué es importante?}

Se ha seleccionado este proyecto y set de datos pues es similar al
estudio que actualmente me encuentro realizando, debido a que me dedico
a medio tiempo a la venta de casas.

\textbf{El objetivo es:}

\begin{itemize}
\item
  Predecir precios justos y competitivos de viviendas de venta.
\item
  Identificar que variables influyen para poder predecir el valor de una
  vivienda
\end{itemize}

\hypertarget{integraciuxf3n-y-selecciuxf3n-de-los-datos-de-interuxe9s-a-analizar.}{%
\subsection{Integración y selección de los datos de interés a
analizar.}\label{integraciuxf3n-y-selecciuxf3n-de-los-datos-de-interuxe9s-a-analizar.}}

Carga de Librerías:

\begin{Shaded}
\begin{Highlighting}[]
\ControlFlowTok{if}\NormalTok{(}\SpecialCharTok{!}\FunctionTok{require}\NormalTok{(ggplot2))\{}
    \FunctionTok{install.packages}\NormalTok{(}\StringTok{\textquotesingle{}ggplot2\textquotesingle{}}\NormalTok{, }\AttributeTok{repos=}\StringTok{\textquotesingle{}http://cran.us.r{-}project.org\textquotesingle{}}\NormalTok{)}
    \FunctionTok{library}\NormalTok{(ggplot2)}
\NormalTok{\}}
\end{Highlighting}
\end{Shaded}

\begin{verbatim}
## Loading required package: ggplot2
\end{verbatim}

\begin{verbatim}
## Warning: package 'ggplot2' was built under R version 4.1.3
\end{verbatim}

\begin{Shaded}
\begin{Highlighting}[]
\ControlFlowTok{if}\NormalTok{(}\SpecialCharTok{!}\FunctionTok{require}\NormalTok{(ggpubr))\{}
    \FunctionTok{install.packages}\NormalTok{(}\StringTok{\textquotesingle{}ggpubr\textquotesingle{}}\NormalTok{, }\AttributeTok{repos=}\StringTok{\textquotesingle{}http://cran.us.r{-}project.org\textquotesingle{}}\NormalTok{)}
    \FunctionTok{library}\NormalTok{(ggpubr)}
\NormalTok{\}}
\end{Highlighting}
\end{Shaded}

\begin{verbatim}
## Loading required package: ggpubr
\end{verbatim}

\begin{verbatim}
## Warning: package 'ggpubr' was built under R version 4.1.3
\end{verbatim}

\begin{Shaded}
\begin{Highlighting}[]
\ControlFlowTok{if}\NormalTok{(}\SpecialCharTok{!}\FunctionTok{require}\NormalTok{(grid))\{}
    \FunctionTok{install.packages}\NormalTok{(}\StringTok{\textquotesingle{}grid\textquotesingle{}}\NormalTok{, }\AttributeTok{repos=}\StringTok{\textquotesingle{}http://cran.us.r{-}project.org\textquotesingle{}}\NormalTok{)}
    \FunctionTok{library}\NormalTok{(grid)}
\NormalTok{\}}
\end{Highlighting}
\end{Shaded}

\begin{verbatim}
## Loading required package: grid
\end{verbatim}

\begin{Shaded}
\begin{Highlighting}[]
\ControlFlowTok{if}\NormalTok{(}\SpecialCharTok{!}\FunctionTok{require}\NormalTok{(gridExtra))\{}
    \FunctionTok{install.packages}\NormalTok{(}\StringTok{\textquotesingle{}gridExtra\textquotesingle{}}\NormalTok{, }\AttributeTok{repos=}\StringTok{\textquotesingle{}http://cran.us.r{-}project.org\textquotesingle{}}\NormalTok{)}
    \FunctionTok{library}\NormalTok{(gridExtra)}
\NormalTok{\}}
\end{Highlighting}
\end{Shaded}

\begin{verbatim}
## Loading required package: gridExtra
\end{verbatim}

\begin{verbatim}
## Warning: package 'gridExtra' was built under R version 4.1.3
\end{verbatim}

\begin{Shaded}
\begin{Highlighting}[]
\ControlFlowTok{if}\NormalTok{(}\SpecialCharTok{!}\FunctionTok{require}\NormalTok{(C50))\{}
    \FunctionTok{install.packages}\NormalTok{(}\StringTok{\textquotesingle{}C50\textquotesingle{}}\NormalTok{, }\AttributeTok{repos=}\StringTok{\textquotesingle{}http://cran.us.r{-}project.org\textquotesingle{}}\NormalTok{)}
    \FunctionTok{library}\NormalTok{(C50)}
\NormalTok{\}}
\end{Highlighting}
\end{Shaded}

\begin{verbatim}
## Loading required package: C50
\end{verbatim}

\begin{verbatim}
## Warning: package 'C50' was built under R version 4.1.3
\end{verbatim}

\begin{Shaded}
\begin{Highlighting}[]
\ControlFlowTok{if}\NormalTok{ (}\SpecialCharTok{!}\FunctionTok{require}\NormalTok{(}\StringTok{\textquotesingle{}papeR\textquotesingle{}}\NormalTok{)) \{}\FunctionTok{install.packages}\NormalTok{(}\StringTok{\textquotesingle{}papeR\textquotesingle{}}\NormalTok{); }\FunctionTok{library}\NormalTok{(}\StringTok{\textquotesingle{}papeR\textquotesingle{}}\NormalTok{)}
\NormalTok{\}}
\end{Highlighting}
\end{Shaded}

\begin{verbatim}
## Loading required package: papeR
\end{verbatim}

\begin{verbatim}
## Warning: package 'papeR' was built under R version 4.1.3
\end{verbatim}

\begin{verbatim}
## Loading required package: car
\end{verbatim}

\begin{verbatim}
## Warning: package 'car' was built under R version 4.1.3
\end{verbatim}

\begin{verbatim}
## Loading required package: carData
\end{verbatim}

\begin{verbatim}
## Warning: package 'carData' was built under R version 4.1.3
\end{verbatim}

\begin{verbatim}
## Loading required package: xtable
\end{verbatim}

\begin{verbatim}
## Warning: package 'xtable' was built under R version 4.1.3
\end{verbatim}

\begin{verbatim}
## Registered S3 method overwritten by 'papeR':
##   method    from
##   Anova.lme car
\end{verbatim}

\begin{verbatim}
## 
## Attaching package: 'papeR'
\end{verbatim}

\begin{verbatim}
## The following object is masked from 'package:utils':
## 
##     toLatex
\end{verbatim}

\begin{Shaded}
\begin{Highlighting}[]
\ControlFlowTok{if}\NormalTok{ (}\SpecialCharTok{!}\FunctionTok{require}\NormalTok{(}\StringTok{\textquotesingle{}corrplot\textquotesingle{}}\NormalTok{)) \{}\FunctionTok{install.packages}\NormalTok{(}\StringTok{\textquotesingle{}corrplot\textquotesingle{}}\NormalTok{); }\FunctionTok{library}\NormalTok{(}\StringTok{\textquotesingle{}corrplot\textquotesingle{}}\NormalTok{)}
\NormalTok{\}}
\end{Highlighting}
\end{Shaded}

\begin{verbatim}
## Loading required package: corrplot
\end{verbatim}

\begin{verbatim}
## Warning: package 'corrplot' was built under R version 4.1.3
\end{verbatim}

\begin{verbatim}
## corrplot 0.92 loaded
\end{verbatim}

\begin{Shaded}
\begin{Highlighting}[]
\ControlFlowTok{if}\NormalTok{(}\SpecialCharTok{!}\FunctionTok{require}\NormalTok{(gmodels))\{}
    \FunctionTok{install.packages}\NormalTok{(}\StringTok{\textquotesingle{}gmodels\textquotesingle{}}\NormalTok{, }\AttributeTok{repos=}\StringTok{\textquotesingle{}http://cran.us.r{-}project.org\textquotesingle{}}\NormalTok{)}
    \FunctionTok{library}\NormalTok{(gmodels)}
\NormalTok{\}}
\end{Highlighting}
\end{Shaded}

\begin{verbatim}
## Loading required package: gmodels
\end{verbatim}

\begin{verbatim}
## Warning: package 'gmodels' was built under R version 4.1.3
\end{verbatim}

\begin{Shaded}
\begin{Highlighting}[]
\ControlFlowTok{if}\NormalTok{ (}\SpecialCharTok{!}\FunctionTok{require}\NormalTok{(}\StringTok{\textquotesingle{}dplyr\textquotesingle{}}\NormalTok{)) \{}\FunctionTok{install.packages}\NormalTok{(}\StringTok{\textquotesingle{}dplyr\textquotesingle{}}\NormalTok{); }\FunctionTok{library}\NormalTok{(}\StringTok{\textquotesingle{}dplyr\textquotesingle{}}\NormalTok{)}
\NormalTok{\}}
\end{Highlighting}
\end{Shaded}

\begin{verbatim}
## Loading required package: dplyr
\end{verbatim}

\begin{verbatim}
## 
## Attaching package: 'dplyr'
\end{verbatim}

\begin{verbatim}
## The following objects are masked from 'package:papeR':
## 
##     summarise, summarize
\end{verbatim}

\begin{verbatim}
## The following object is masked from 'package:car':
## 
##     recode
\end{verbatim}

\begin{verbatim}
## The following object is masked from 'package:gridExtra':
## 
##     combine
\end{verbatim}

\begin{verbatim}
## The following objects are masked from 'package:stats':
## 
##     filter, lag
\end{verbatim}

\begin{verbatim}
## The following objects are masked from 'package:base':
## 
##     intersect, setdiff, setequal, union
\end{verbatim}

\begin{Shaded}
\begin{Highlighting}[]
\ControlFlowTok{if}\NormalTok{ (}\SpecialCharTok{!}\FunctionTok{require}\NormalTok{(}\StringTok{\textquotesingle{}stringr\textquotesingle{}}\NormalTok{)) \{}\FunctionTok{install.packages}\NormalTok{(}\StringTok{\textquotesingle{}stringr\textquotesingle{}}\NormalTok{); }\FunctionTok{library}\NormalTok{(}\StringTok{\textquotesingle{}stringr\textquotesingle{}}\NormalTok{)}
\NormalTok{\}}
\end{Highlighting}
\end{Shaded}

\begin{verbatim}
## Loading required package: stringr
\end{verbatim}

\begin{Shaded}
\begin{Highlighting}[]
\ControlFlowTok{if}\NormalTok{ (}\SpecialCharTok{!}\FunctionTok{require}\NormalTok{(}\StringTok{\textquotesingle{}nortest\textquotesingle{}}\NormalTok{)) \{}\FunctionTok{install.packages}\NormalTok{(}\StringTok{\textquotesingle{}nortest\textquotesingle{}}\NormalTok{); }\FunctionTok{library}\NormalTok{(}\StringTok{\textquotesingle{}nortest\textquotesingle{}}\NormalTok{)}
\NormalTok{\}}
\end{Highlighting}
\end{Shaded}

\begin{verbatim}
## Loading required package: nortest
\end{verbatim}

\begin{Shaded}
\begin{Highlighting}[]
\ControlFlowTok{if}\NormalTok{ (}\SpecialCharTok{!}\FunctionTok{require}\NormalTok{(}\StringTok{\textquotesingle{}scales\textquotesingle{}}\NormalTok{)) \{}\FunctionTok{install.packages}\NormalTok{(}\StringTok{\textquotesingle{}scales\textquotesingle{}}\NormalTok{); }\FunctionTok{library}\NormalTok{(}\StringTok{\textquotesingle{}scales\textquotesingle{}}\NormalTok{)}
\NormalTok{\}}
\end{Highlighting}
\end{Shaded}

\begin{verbatim}
## Loading required package: scales
\end{verbatim}

\begin{verbatim}
## Warning: package 'scales' was built under R version 4.1.3
\end{verbatim}

\begin{Shaded}
\begin{Highlighting}[]
\ControlFlowTok{if}\NormalTok{ (}\SpecialCharTok{!}\FunctionTok{require}\NormalTok{(}\StringTok{\textquotesingle{}broom\textquotesingle{}}\NormalTok{)) \{}\FunctionTok{install.packages}\NormalTok{(}\StringTok{\textquotesingle{}broom\textquotesingle{}}\NormalTok{); }\FunctionTok{library}\NormalTok{(}\StringTok{\textquotesingle{}broom\textquotesingle{}}\NormalTok{)}
\NormalTok{\}}
\end{Highlighting}
\end{Shaded}

\begin{verbatim}
## Loading required package: broom
\end{verbatim}

\begin{verbatim}
## Warning: package 'broom' was built under R version 4.1.3
\end{verbatim}

Carga de los datos:

\begin{Shaded}
\begin{Highlighting}[]
\NormalTok{data\_set}\OtherTok{\textless{}{-}}\FunctionTok{read.csv}\NormalTok{(}\StringTok{"./data.csv"}\NormalTok{,}\AttributeTok{header=}\NormalTok{T,}\AttributeTok{sep=}\StringTok{","}\NormalTok{)}
\FunctionTok{attach}\NormalTok{(data\_set)}
\end{Highlighting}
\end{Shaded}

\hypertarget{limpieza-de-los-datos.}{%
\subsection{Limpieza de los datos.}\label{limpieza-de-los-datos.}}

Para empezar, calculamos las dimensiones de la base de datos mediante la
función str(). Obtenemos que disponemos de 33656 registros (filas) y 19
variables (columnas).

Se procede a eliminar los atributos que no aportan a nuestro estudio,
entre estos están atributos con valores tipo texto imposibles de
categorizar(NEAREST\_STN,ADDRESS,NEAREST\_SCH,DATE\_SOLD) y atributos no
necesarios(LONGITUDE, LATITUDE,POSTCODE).

\begin{Shaded}
\begin{Highlighting}[]
\NormalTok{data\_set[}\FunctionTok{c}\NormalTok{(}\StringTok{"NEAREST\_STN"}\NormalTok{,}\StringTok{"ADDRESS"}\NormalTok{,}\StringTok{"NEAREST\_SCH"}\NormalTok{,}\StringTok{"DATE\_SOLD"}\NormalTok{,}\StringTok{"LONGITUDE"}\NormalTok{,}\StringTok{"LATITUDE"}\NormalTok{, }\StringTok{"POSTCODE"}\NormalTok{)] }\OtherTok{\textless{}{-}} \ConstantTok{NULL}
\end{Highlighting}
\end{Shaded}

Se puede observar que existen variables con diferente tipo de dato, por
ejemplo garage ,buil\_year deberían ser del tipo numérico.

\begin{Shaded}
\begin{Highlighting}[]
\FunctionTok{str}\NormalTok{(data\_set)}
\end{Highlighting}
\end{Shaded}

\begin{verbatim}
## 'data.frame':    33656 obs. of  12 variables:
##  $ SUBURB          : chr  "South Lake" "Wandi" "Camillo" "Bellevue" ...
##  $ PRICE           : int  565000 365000 287000 255000 325000 409000 400000 370000 565000 685000 ...
##  $ BEDROOMS        : int  4 3 3 2 4 4 3 4 4 3 ...
##  $ BATHROOMS       : int  2 2 1 1 1 2 2 2 2 2 ...
##  $ GARAGE          : chr  "2" "2" "1" "2" ...
##  $ LAND_AREA       : int  600 351 719 651 466 759 386 468 875 552 ...
##  $ FLOOR_AREA      : int  160 139 86 59 131 118 132 158 168 126 ...
##  $ BUILD_YEAR      : chr  "2003" "2013" "1979" "1953" ...
##  $ CBD_DIST        : int  18300 26900 22600 17900 11200 27300 28200 41700 12100 5900 ...
##  $ NEAREST_STN_DIST: int  1800 4900 1900 3600 2000 1000 3700 1100 2500 508 ...
##  $ NEAREST_SCH_DIST: num  0.828 5.524 1.649 1.571 1.515 ...
##  $ NEAREST_SCH_RANK: int  NA 129 113 NA NA NA NA NA NA 29 ...
\end{verbatim}

Remplazamos los valores en letras NULL por NA y cambiamos el tipo de
dato para las columnas buil\_year, garage a numéricas

\begin{Shaded}
\begin{Highlighting}[]
\NormalTok{data\_set[data\_set }\SpecialCharTok{==} \StringTok{"NULL"}\NormalTok{]}\OtherTok{\textless{}{-}} \ConstantTok{NA}
\NormalTok{data\_set}\SpecialCharTok{$}\NormalTok{BUILD\_YEAR}\OtherTok{=}\FunctionTok{as.numeric}\NormalTok{(data\_set}\SpecialCharTok{$}\NormalTok{BUILD\_YEAR)}
\NormalTok{data\_set}\SpecialCharTok{$}\NormalTok{GARAGE}\OtherTok{=}\FunctionTok{as.numeric}\NormalTok{(data\_set}\SpecialCharTok{$}\NormalTok{GARAGE)}
\NormalTok{data\_set}\SpecialCharTok{$}\NormalTok{LAND\_AREA}\OtherTok{=}\FunctionTok{as.numeric}\NormalTok{(data\_set}\SpecialCharTok{$}\NormalTok{LAND\_AREA)}
\NormalTok{data\_set}\SpecialCharTok{$}\NormalTok{FLOOR\_AREA}\OtherTok{=}\FunctionTok{as.numeric}\NormalTok{(data\_set}\SpecialCharTok{$}\NormalTok{FLOOR\_AREA)}
\NormalTok{data\_set}\SpecialCharTok{$}\NormalTok{PRICE}\OtherTok{=}\FunctionTok{as.numeric}\NormalTok{(data\_set}\SpecialCharTok{$}\NormalTok{PRICE)}
\end{Highlighting}
\end{Shaded}

Se considerá crear las siguientes variables:

\begin{itemize}
\tightlist
\item
  Una variable sobre el Precio/M2 (superficie terreno total).
\end{itemize}

\begin{Shaded}
\begin{Highlighting}[]
\NormalTok{data\_set}\SpecialCharTok{$}\NormalTok{Precio\_M2\_total}\OtherTok{\textless{}{-}}\NormalTok{ (data\_set}\SpecialCharTok{$}\NormalTok{PRICE}\SpecialCharTok{/}\NormalTok{data\_set}\SpecialCharTok{$}\NormalTok{LAND\_AREA)}
\end{Highlighting}
\end{Shaded}

Al crear las nuevas variables, se procede a eliminar las variables:
LAND\_AREA,FLOOR\_AREA,PRICE, debido a que se estaría redundando la
información.

\begin{Shaded}
\begin{Highlighting}[]
\NormalTok{data\_set[}\FunctionTok{c}\NormalTok{(}\StringTok{"LAND\_AREA"}\NormalTok{,}\StringTok{"FLOOR\_AREA"}\NormalTok{,}\StringTok{"BATHROOMS"}\NormalTok{,}\StringTok{"BEDROOMS"}\NormalTok{,}\StringTok{"GARAGE"}\NormalTok{)] }\OtherTok{\textless{}{-}} \ConstantTok{NULL}
\end{Highlighting}
\end{Shaded}

Se identifica que las variables
NEAREST\_STN\_DIST,NEAREST\_SCH\_DIST,CBD\_DIST no estan en la misma
escala, por lo que se procede a cambiar las variables
NEAREST\_STN\_DIST,CBD\_DIST dividiendo para 1000

\begin{Shaded}
\begin{Highlighting}[]
\NormalTok{data\_set}\SpecialCharTok{$}\NormalTok{NEAREST\_STN\_DIST}\OtherTok{\textless{}{-}}\NormalTok{ (data\_set}\SpecialCharTok{$}\NormalTok{NEAREST\_STN\_DIST}\SpecialCharTok{/}\DecValTok{1000}\NormalTok{)}
\NormalTok{data\_set}\SpecialCharTok{$}\NormalTok{CBD\_DIST}\OtherTok{\textless{}{-}}\NormalTok{ (data\_set}\SpecialCharTok{$}\NormalTok{CBD\_DIST}\SpecialCharTok{/}\DecValTok{1000}\NormalTok{)}
\NormalTok{data\_set}\SpecialCharTok{$}\NormalTok{PRICE}\OtherTok{\textless{}{-}}\NormalTok{ (data\_set}\SpecialCharTok{$}\NormalTok{PRICE}\SpecialCharTok{/}\DecValTok{1000}\NormalTok{)}
\end{Highlighting}
\end{Shaded}

\begin{Shaded}
\begin{Highlighting}[]
\FunctionTok{str}\NormalTok{(data\_set)}
\end{Highlighting}
\end{Shaded}

\begin{verbatim}
## 'data.frame':    33656 obs. of  8 variables:
##  $ SUBURB          : chr  "South Lake" "Wandi" "Camillo" "Bellevue" ...
##  $ PRICE           : num  565 365 287 255 325 409 400 370 565 685 ...
##  $ BUILD_YEAR      : num  2003 2013 1979 1953 1998 ...
##  $ CBD_DIST        : num  18.3 26.9 22.6 17.9 11.2 27.3 28.2 41.7 12.1 5.9 ...
##  $ NEAREST_STN_DIST: num  1.8 4.9 1.9 3.6 2 1 3.7 1.1 2.5 0.508 ...
##  $ NEAREST_SCH_DIST: num  0.828 5.524 1.649 1.571 1.515 ...
##  $ NEAREST_SCH_RANK: int  NA 129 113 NA NA NA NA NA NA 29 ...
##  $ Precio_M2_total : num  942 1040 399 392 697 ...
\end{verbatim}

\hypertarget{los-datos-contienen-ceros-o-elementos-vacuxedos}{%
\subsubsection{¿Los datos contienen ceros o elementos
vacíos?}\label{los-datos-contienen-ceros-o-elementos-vacuxedos}}

Mostraremos para cada atributo la cantidad de valores perdidos mediante
la función summary. Se identifica que si existen valores nulos para las
columnas GARAGE, BUILD\_YEAR y NEAREST\_SCH\_RANK. No existen valores
vacios.

\begin{Shaded}
\begin{Highlighting}[]
\NormalTok{NAcount }\OtherTok{\textless{}{-}} \FunctionTok{apply}\NormalTok{(}\FunctionTok{is.na}\NormalTok{(data\_set), }\DecValTok{2}\NormalTok{, sum) }\SpecialCharTok{\textgreater{}=} \DecValTok{1}
\NormalTok{NAcount[NAcount}\SpecialCharTok{==}\ConstantTok{TRUE}\NormalTok{]}
\end{Highlighting}
\end{Shaded}

\begin{verbatim}
##       BUILD_YEAR NEAREST_SCH_RANK 
##             TRUE             TRUE
\end{verbatim}

\begin{Shaded}
\begin{Highlighting}[]
\NormalTok{WScount }\OtherTok{\textless{}{-}} \FunctionTok{colSums}\NormalTok{(data\_set }\SpecialCharTok{==} \StringTok{""}\NormalTok{)}
\NormalTok{WScount[WScount}\SpecialCharTok{==}\ConstantTok{TRUE}\NormalTok{]}
\end{Highlighting}
\end{Shaded}

\begin{verbatim}
## <NA> <NA> 
##   NA   NA
\end{verbatim}

Como parte de la preparación de los datos, miraremos si hay valores
missing .

\begin{Shaded}
\begin{Highlighting}[]
\NormalTok{missing }\OtherTok{\textless{}{-}}\NormalTok{ data\_set[}\FunctionTok{is.na}\NormalTok{(data\_set),]}
\FunctionTok{dim}\NormalTok{(missing)}
\end{Highlighting}
\end{Shaded}

\begin{verbatim}
## [1] 14107     8
\end{verbatim}

Observamos fácilmente que si hay valores missing y, por tanto, deberemos
preparar los datos en este sentido.

\textbf{Imputación de valores para el atributo ``BUILD\_YEAR''}

Reemplazamos los valores missing (na), por la media, porque si no tiene
información nos inclinamos obtener un promedio de los años del resto de
registros, pues colocar un valor 0 se convertiría en un outlier

\begin{Shaded}
\begin{Highlighting}[]
\NormalTok{data\_set}\SpecialCharTok{$}\NormalTok{BUILD\_YEAR[}\FunctionTok{is.na}\NormalTok{(data\_set}\SpecialCharTok{$}\NormalTok{BUILD\_YEAR)]}\OtherTok{\textless{}{-}}\FunctionTok{mean}\NormalTok{(data\_set}\SpecialCharTok{$}\NormalTok{BUILD\_YEAR, }\AttributeTok{na.rm =}\NormalTok{ T)}
\NormalTok{data\_set}\SpecialCharTok{$}\NormalTok{BUILD\_YEAR}\OtherTok{\textless{}{-}}\FunctionTok{round}\NormalTok{(data\_set}\SpecialCharTok{$}\NormalTok{BUILD\_YEAR, }\AttributeTok{digits =} \DecValTok{0}\NormalTok{)}
\end{Highlighting}
\end{Shaded}

\textbf{Imputación de valores para el atributo ``NEAREST\_SCH\_RANK''}
Se identifica que 10952, registros son nulos de un total 33656, lo cual
corresponde a mas del 30\% por ese motivo se procede a eliminar la
columna de nuestro data\_set

\begin{Shaded}
\begin{Highlighting}[]
\FunctionTok{summary}\NormalTok{ (data\_set}\SpecialCharTok{$}\NormalTok{NEAREST\_SCH\_RANK)}
\end{Highlighting}
\end{Shaded}

\begin{verbatim}
##    Min. 1st Qu.  Median    Mean 3rd Qu.    Max.    NA's 
##    1.00   39.00   68.00   72.67  105.00  139.00   10952
\end{verbatim}

\begin{Shaded}
\begin{Highlighting}[]
\NormalTok{data\_set[}\FunctionTok{c}\NormalTok{(}\StringTok{"NEAREST\_SCH\_RANK"}\NormalTok{)] }\OtherTok{\textless{}{-}} \ConstantTok{NULL}
\end{Highlighting}
\end{Shaded}

\hypertarget{identifica-y-gestiona-los-valores-extremos.}{%
\subsubsection{Identifica y gestiona los valores
extremos.}\label{identifica-y-gestiona-los-valores-extremos.}}

Se muestra las estadísticas de las diferentes variables.

\begin{Shaded}
\begin{Highlighting}[]
\FunctionTok{options}\NormalTok{(}\AttributeTok{width =} \DecValTok{90}\NormalTok{)}
\NormalTok{papeR}\SpecialCharTok{::}\FunctionTok{summarize}\NormalTok{(data\_set)}
\end{Highlighting}
\end{Shaded}

\begin{verbatim}
## Factors are dropped from the summary
\end{verbatim}

\begin{verbatim}
##                        N      Mean     SD       Min      Q1  Median      Q3     Max
## 1            PRICE 33656    637.07 355.83     51.00  410.00  535.50  760.00 2440.00
## 2       BUILD_YEAR 33656   1989.73  19.96   1868.00 1980.00 1993.00 2004.00 2017.00
## 3         CBD_DIST 33656     19.78  11.36      0.68   11.20   17.50   26.60   59.80
## 4 NEAREST_STN_DIST 33656      4.52   4.50      0.05    1.80    3.20    5.30   35.50
## 5 NEAREST_SCH_DIST 33656      1.82   1.75      0.07    0.88    1.35    2.10   23.25
## 6  Precio_M2_total 33656    998.77 802.31      0.37  509.05  824.78 1248.09 9944.75
\end{verbatim}

Se identifica que existen valores extremos sin embargo pueden ser
valores reales los cuales dependan del lugar en el que se encuentren.

\textbf{Valores atípicos} Se realiza una representación en diagrama de
cajas para las variables numéricas. Para la identificación de los
valores atípico se hace uso de una gráfica tipo box-plot, en donde se
observan los valores basados en cuartiles.

Variable CBD\_DIST: Para aquellos valores atípicos identificados se
reemplaza por el valor NaN, luego estos serán analizados para
identificar si se eliminan o se realiza un proceso de regresión para
colocar el valor.

\begin{Shaded}
\begin{Highlighting}[]
\NormalTok{valoresAtipicos }\OtherTok{\textless{}{-}} \FunctionTok{boxplot.stats}\NormalTok{(data\_set}\SpecialCharTok{$}\NormalTok{CBD\_DIST)}\SpecialCharTok{$}\NormalTok{out}
\FunctionTok{print}\NormalTok{(}\StringTok{"El tamaño de valores atípicos es:"}\NormalTok{)}
\end{Highlighting}
\end{Shaded}

\begin{verbatim}
## [1] "El tamaño de valores atípicos es:"
\end{verbatim}

\begin{Shaded}
\begin{Highlighting}[]
\FunctionTok{length}\NormalTok{(valoresAtipicos)}
\end{Highlighting}
\end{Shaded}

\begin{verbatim}
## [1] 661
\end{verbatim}

\begin{Shaded}
\begin{Highlighting}[]
\NormalTok{data\_set}\SpecialCharTok{$}\NormalTok{CBD\_DIST[data\_set}\SpecialCharTok{$}\NormalTok{CBD\_DIST }\SpecialCharTok{\%in\%}\NormalTok{ valoresAtipicos] }\OtherTok{\textless{}{-}} \ConstantTok{NaN}
\end{Highlighting}
\end{Shaded}

Variable BUILD\_YEAR: Para aquellos valores atípicos identificados se
reemplaza por el valor NaN, luego estos serán analizados para
identificar si se eliminan o se realiza un proceso de regresión para
colocar el valor.

\begin{Shaded}
\begin{Highlighting}[]
\NormalTok{valoresAtipicos }\OtherTok{\textless{}{-}} \FunctionTok{boxplot.stats}\NormalTok{(data\_set}\SpecialCharTok{$}\NormalTok{BUILD\_YEAR)}\SpecialCharTok{$}\NormalTok{out}
\FunctionTok{print}\NormalTok{(}\StringTok{"El tamaño de valores atípicos es:"}\NormalTok{)}
\end{Highlighting}
\end{Shaded}

\begin{verbatim}
## [1] "El tamaño de valores atípicos es:"
\end{verbatim}

\begin{Shaded}
\begin{Highlighting}[]
\FunctionTok{length}\NormalTok{(valoresAtipicos)}
\end{Highlighting}
\end{Shaded}

\begin{verbatim}
## [1] 1183
\end{verbatim}

\begin{Shaded}
\begin{Highlighting}[]
\NormalTok{data\_set}\SpecialCharTok{$}\NormalTok{BUILD\_YEAR[data\_set}\SpecialCharTok{$}\NormalTok{BUILD\_YEAR }\SpecialCharTok{\%in\%}\NormalTok{ valoresAtipicos] }\OtherTok{\textless{}{-}} \ConstantTok{NaN}
\end{Highlighting}
\end{Shaded}

Variable NEAREST\_STN\_DIST: Para aquellos valores atípicos
identificados se reemplaza por el valor NaN, luego estos serán
analizados para identificar si se eliminan o se realiza un proceso de
regresión para colocar el valor.

\begin{Shaded}
\begin{Highlighting}[]
\NormalTok{valoresAtipicos }\OtherTok{\textless{}{-}} \FunctionTok{boxplot.stats}\NormalTok{(data\_set}\SpecialCharTok{$}\NormalTok{NEAREST\_STN\_DIST)}\SpecialCharTok{$}\NormalTok{out}
\FunctionTok{print}\NormalTok{(}\StringTok{"El tamaño de valores atípicos es:"}\NormalTok{)}
\end{Highlighting}
\end{Shaded}

\begin{verbatim}
## [1] "El tamaño de valores atípicos es:"
\end{verbatim}

\begin{Shaded}
\begin{Highlighting}[]
\FunctionTok{length}\NormalTok{(valoresAtipicos)}
\end{Highlighting}
\end{Shaded}

\begin{verbatim}
## [1] 3034
\end{verbatim}

\begin{Shaded}
\begin{Highlighting}[]
\NormalTok{data\_set}\SpecialCharTok{$}\NormalTok{NEAREST\_STN\_DIST[data\_set}\SpecialCharTok{$}\NormalTok{NEAREST\_STN\_DIST }\SpecialCharTok{\%in\%}\NormalTok{ valoresAtipicos] }\OtherTok{\textless{}{-}} \ConstantTok{NaN}
\end{Highlighting}
\end{Shaded}

Variable NEAREST\_SCH\_DIST: Para aquellos valores atípicos
identificados se reemplaza por el valor NaN, luego estos serán
analizados para identificar si se eliminan o se realiza un proceso de
regresión para colocar el valor.

\begin{Shaded}
\begin{Highlighting}[]
\NormalTok{valoresAtipicos }\OtherTok{\textless{}{-}} \FunctionTok{boxplot.stats}\NormalTok{(data\_set}\SpecialCharTok{$}\NormalTok{NEAREST\_SCH\_DIST)}\SpecialCharTok{$}\NormalTok{out}
\FunctionTok{print}\NormalTok{(}\StringTok{"El tamaño de valores atípicos es:"}\NormalTok{)}
\end{Highlighting}
\end{Shaded}

\begin{verbatim}
## [1] "El tamaño de valores atípicos es:"
\end{verbatim}

\begin{Shaded}
\begin{Highlighting}[]
\FunctionTok{length}\NormalTok{(valoresAtipicos)}
\end{Highlighting}
\end{Shaded}

\begin{verbatim}
## [1] 2296
\end{verbatim}

\begin{Shaded}
\begin{Highlighting}[]
\NormalTok{data\_set}\SpecialCharTok{$}\NormalTok{NEAREST\_SCH\_DIST[data\_set}\SpecialCharTok{$}\NormalTok{NEAREST\_SCH\_DIST }\SpecialCharTok{\%in\%}\NormalTok{ valoresAtipicos] }\OtherTok{\textless{}{-}} \ConstantTok{NaN}
\end{Highlighting}
\end{Shaded}

Variable Precio\_M2\_total: Para aquellos valores atípicos identificados
se reemplaza por el valor NaN, luego estos serán analizados para
identificar si se eliminan o se realiza un proceso de regresión para
colocar el valor.

\begin{Shaded}
\begin{Highlighting}[]
\NormalTok{valoresAtipicos }\OtherTok{\textless{}{-}} \FunctionTok{boxplot.stats}\NormalTok{(data\_set}\SpecialCharTok{$}\NormalTok{Precio\_M2\_total)}\SpecialCharTok{$}\NormalTok{out}
\FunctionTok{print}\NormalTok{(}\StringTok{"El tamaño de valores atípicos es:"}\NormalTok{)}
\end{Highlighting}
\end{Shaded}

\begin{verbatim}
## [1] "El tamaño de valores atípicos es:"
\end{verbatim}

\begin{Shaded}
\begin{Highlighting}[]
\FunctionTok{length}\NormalTok{(valoresAtipicos)}
\end{Highlighting}
\end{Shaded}

\begin{verbatim}
## [1] 2113
\end{verbatim}

\begin{Shaded}
\begin{Highlighting}[]
\NormalTok{data\_set}\SpecialCharTok{$}\NormalTok{Precio\_M2\_total[data\_set}\SpecialCharTok{$}\NormalTok{Precio\_M2\_total }\SpecialCharTok{\%in\%}\NormalTok{ valoresAtipicos] }\OtherTok{\textless{}{-}} \ConstantTok{NaN}
\end{Highlighting}
\end{Shaded}

Variable PRICE: Para aquellos valores atípicos identificados se
reemplaza por el valor NaN, luego estos serán analizados para
identificar si se eliminan o se realiza un proceso de regresión para
colocar el valor.

\begin{Shaded}
\begin{Highlighting}[]
\NormalTok{valoresAtipicos }\OtherTok{\textless{}{-}} \FunctionTok{boxplot.stats}\NormalTok{(data\_set}\SpecialCharTok{$}\NormalTok{PRICE)}\SpecialCharTok{$}\NormalTok{out}
\FunctionTok{print}\NormalTok{(}\StringTok{"El tamaño de valores atípicos es:"}\NormalTok{)}
\end{Highlighting}
\end{Shaded}

\begin{verbatim}
## [1] "El tamaño de valores atípicos es:"
\end{verbatim}

\begin{Shaded}
\begin{Highlighting}[]
\FunctionTok{length}\NormalTok{(valoresAtipicos)}
\end{Highlighting}
\end{Shaded}

\begin{verbatim}
## [1] 2112
\end{verbatim}

\begin{Shaded}
\begin{Highlighting}[]
\NormalTok{data\_set}\SpecialCharTok{$}\NormalTok{PRICE[data\_set}\SpecialCharTok{$}\NormalTok{PRICE }\SpecialCharTok{\%in\%}\NormalTok{ valoresAtipicos] }\OtherTok{\textless{}{-}} \ConstantTok{NaN}
\end{Highlighting}
\end{Shaded}

Calculo de valores atipicos, se llega a la conclusión que estos
registros estan mal registrados

\begin{Shaded}
\begin{Highlighting}[]
\NormalTok{data\_set }\OtherTok{\textless{}{-}}\NormalTok{ data\_set[}\SpecialCharTok{!}\FunctionTok{is.na}\NormalTok{(data\_set}\SpecialCharTok{$}\NormalTok{BUILD\_YEAR),]}
\NormalTok{data\_set }\OtherTok{\textless{}{-}}\NormalTok{ data\_set[}\SpecialCharTok{!}\FunctionTok{is.na}\NormalTok{(data\_set}\SpecialCharTok{$}\NormalTok{CBD\_DIST),]}
\NormalTok{data\_set }\OtherTok{\textless{}{-}}\NormalTok{ data\_set[}\SpecialCharTok{!}\FunctionTok{is.na}\NormalTok{(data\_set}\SpecialCharTok{$}\NormalTok{NEAREST\_STN\_DIST),]}
\NormalTok{data\_set }\OtherTok{\textless{}{-}}\NormalTok{ data\_set[}\SpecialCharTok{!}\FunctionTok{is.na}\NormalTok{(data\_set}\SpecialCharTok{$}\NormalTok{NEAREST\_SCH\_DIST),]}
\NormalTok{data\_set }\OtherTok{\textless{}{-}}\NormalTok{ data\_set[}\SpecialCharTok{!}\FunctionTok{is.na}\NormalTok{(data\_set}\SpecialCharTok{$}\NormalTok{Precio\_M2\_total),]}
\NormalTok{data\_set }\OtherTok{\textless{}{-}}\NormalTok{ data\_set[}\SpecialCharTok{!}\FunctionTok{is.na}\NormalTok{(data\_set}\SpecialCharTok{$}\NormalTok{PRICE),]}
\end{Highlighting}
\end{Shaded}

\hypertarget{anuxe1lisis-de-los-datos.}{%
\subsection{Análisis de los datos.}\label{anuxe1lisis-de-los-datos.}}

\textbf{Métodos de discretización}

BUILD\_YEAR,existe una marcada diferencia entre la mediana y la media
luego se puede presumir que existen valores atípicos en los datos.
Adicionalmente la variable no sigue una distribución normal.

\begin{Shaded}
\begin{Highlighting}[]
\NormalTok{xplot }\OtherTok{\textless{}{-}} \FunctionTok{ggdensity}\NormalTok{(data\_set, }\StringTok{"BUILD\_YEAR"}\NormalTok{, }\AttributeTok{fill =} \StringTok{"BUILD\_YEAR"}\NormalTok{,}
                   \AttributeTok{palette =} \StringTok{"jco"}\NormalTok{)}\SpecialCharTok{+}
  \FunctionTok{stat\_overlay\_normal\_density}\NormalTok{(}\AttributeTok{color =} \StringTok{"red"}\NormalTok{, }\AttributeTok{linetype =} \StringTok{"dashed"}\NormalTok{)}
\FunctionTok{ggarrange}\NormalTok{(xplot,}\AttributeTok{nrow =} \DecValTok{1}\NormalTok{, }\AttributeTok{labels =} \StringTok{"Distribución"}\NormalTok{) }
\end{Highlighting}
\end{Shaded}

\includegraphics{sguaigua_PRA2_files/figure-latex/unnamed-chunk-22-1.pdf}

NEAREST\_SCH\_DIST, existe una marcada diferencia entre la mediana y la
media luego se puede presumir que existen valores atípicos en los datos.
Adicionalmente la variable no sigue una distribución normal.

\begin{Shaded}
\begin{Highlighting}[]
\NormalTok{xplot }\OtherTok{\textless{}{-}} \FunctionTok{ggdensity}\NormalTok{(data\_set, }\StringTok{"NEAREST\_SCH\_DIST"}\NormalTok{, }\AttributeTok{fill =} \StringTok{"NEAREST\_SCH\_DIST"}\NormalTok{,}
                   \AttributeTok{palette =} \StringTok{"jco"}\NormalTok{)}\SpecialCharTok{+}
  \FunctionTok{stat\_overlay\_normal\_density}\NormalTok{(}\AttributeTok{color =} \StringTok{"red"}\NormalTok{, }\AttributeTok{linetype =} \StringTok{"dashed"}\NormalTok{)}
\FunctionTok{ggarrange}\NormalTok{(xplot,}\AttributeTok{nrow =} \DecValTok{1}\NormalTok{, }\AttributeTok{labels =} \StringTok{"Distribución"}\NormalTok{) }
\end{Highlighting}
\end{Shaded}

\includegraphics{sguaigua_PRA2_files/figure-latex/unnamed-chunk-23-1.pdf}
Precio\_M2\_total, existe una marcada diferencia entre la mediana y la
media luego se puede presumir que existen valores atípicos en los datos.
Adicionalmente la variable no sigue una distribución normal.

\begin{Shaded}
\begin{Highlighting}[]
\NormalTok{xplot }\OtherTok{\textless{}{-}} \FunctionTok{ggdensity}\NormalTok{(data\_set, }\StringTok{"PRICE"}\NormalTok{, }\AttributeTok{fill =} \StringTok{"PRICE"}\NormalTok{,}
                   \AttributeTok{palette =} \StringTok{"jco"}\NormalTok{)}\SpecialCharTok{+}
  \FunctionTok{stat\_overlay\_normal\_density}\NormalTok{(}\AttributeTok{color =} \StringTok{"red"}\NormalTok{, }\AttributeTok{linetype =} \StringTok{"dashed"}\NormalTok{)}
\FunctionTok{ggarrange}\NormalTok{(xplot,}\AttributeTok{nrow =} \DecValTok{1}\NormalTok{, }\AttributeTok{labels =} \StringTok{"Distribución"}\NormalTok{) }
\end{Highlighting}
\end{Shaded}

\includegraphics{sguaigua_PRA2_files/figure-latex/unnamed-chunk-24-1.pdf}

En el conjunto de datos analizado se puede observar las variables
continuas BUILD\_YEAR
CBD\_DIST,NEAREST\_STN\_DIST,NEAREST\_SCH\_DIST,Precio\_M2\_total, son
candidatas a realizar procesos de discretización para generar nuevos
clústeres que puedan representar los datos. Se seleccionar estas
variables porque su estructura puede ser representada como rangos de
clústeres.

Para esto se introduce nuevas variables SEG\_BUILD\_YEAR
SEG\_CBD\_DIST,SEG\_NEAREST\_STN\_DIST,SEG\_NEAREST\_SCH\_DIST,SEG\_Precio\_M2\_total,
en donde los rangos tienen la misma amplitud debido a que para ciertos
casos no hay muchos registros con esas condiciones. La discretización de
datos nos permitirá utilizarlos posteriormente en los análisis visuales.

\begin{Shaded}
\begin{Highlighting}[]
\NormalTok{data\_set[}\StringTok{"SEG\_BUILD\_YEAR"}\NormalTok{] }\OtherTok{\textless{}{-}} 
  \FunctionTok{cut}\NormalTok{(data\_set}\SpecialCharTok{$}\NormalTok{BUILD\_YEAR, }\AttributeTok{breaks =} \FunctionTok{c}\NormalTok{(}\DecValTok{0}\NormalTok{,}\DecValTok{1950}\NormalTok{,}\DecValTok{1975}\NormalTok{,}\DecValTok{2000}\NormalTok{,}\DecValTok{2010}\NormalTok{,}\DecValTok{2017}\NormalTok{), }
      \AttributeTok{labels =} \FunctionTok{c}\NormalTok{(}\StringTok{"50\textquotesingle{}s"}\NormalTok{,}\StringTok{"80\textquotesingle{}s"}\NormalTok{,}\StringTok{"2000"}\NormalTok{,}\StringTok{"2010"}\NormalTok{,}\StringTok{"2015\textgreater{}="}\NormalTok{))}

\NormalTok{data\_set[}\StringTok{"SEG\_CBD\_DIST"}\NormalTok{] }\OtherTok{\textless{}{-}} 
  \FunctionTok{cut}\NormalTok{(data\_set}\SpecialCharTok{$}\NormalTok{CBD\_DIST, }\AttributeTok{breaks =} \FunctionTok{c}\NormalTok{(}\DecValTok{0}\NormalTok{,}\DecValTok{2}\NormalTok{,}\DecValTok{10}\NormalTok{,}\DecValTok{20}\NormalTok{,}\DecValTok{30}\NormalTok{,}\DecValTok{40}\NormalTok{,}\DecValTok{50}\NormalTok{,}\DecValTok{60}\NormalTok{), }
      \AttributeTok{labels =} \FunctionTok{c}\NormalTok{(}\StringTok{"\textless{}=2"}\NormalTok{,}\StringTok{"3{-}9"}\NormalTok{,}\StringTok{"10{-}19"}\NormalTok{,}\StringTok{"20{-}29"}\NormalTok{,}\StringTok{"30{-}39"}\NormalTok{,}\StringTok{"40{-}49"}\NormalTok{,}\StringTok{"50\textgreater{}="}\NormalTok{))}
\NormalTok{data\_set[}\StringTok{"SEG\_NEAREST\_STN\_DIST"}\NormalTok{] }\OtherTok{\textless{}{-}} 
  \FunctionTok{cut}\NormalTok{(data\_set}\SpecialCharTok{$}\NormalTok{NEAREST\_STN\_DIST, }\AttributeTok{breaks =} \FunctionTok{c}\NormalTok{(}\DecValTok{0}\NormalTok{,}\DecValTok{1}\NormalTok{,}\DecValTok{3}\NormalTok{,}\DecValTok{5}\NormalTok{,}\DecValTok{10}\NormalTok{,}\DecValTok{25}\NormalTok{,}\DecValTok{35}\NormalTok{), }
      \AttributeTok{labels =} \FunctionTok{c}\NormalTok{(}\StringTok{"\textless{}=1"}\NormalTok{,}\StringTok{"2{-}3"}\NormalTok{,}\StringTok{"4{-}5"}\NormalTok{,}\StringTok{"6{-}10"}\NormalTok{,}\StringTok{"11{-}25"}\NormalTok{,}\StringTok{"26\textgreater{}="}\NormalTok{))}
\NormalTok{data\_set[}\StringTok{"SEG\_NEAREST\_SCH\_DIST"}\NormalTok{] }\OtherTok{\textless{}{-}} 
  \FunctionTok{cut}\NormalTok{(data\_set}\SpecialCharTok{$}\NormalTok{NEAREST\_SCH\_DIST, }\AttributeTok{breaks =} \FunctionTok{c}\NormalTok{(}\DecValTok{0}\NormalTok{,}\DecValTok{1}\NormalTok{,}\DecValTok{3}\NormalTok{,}\DecValTok{5}\NormalTok{,}\DecValTok{10}\NormalTok{,}\DecValTok{25}\NormalTok{,}\DecValTok{50}\NormalTok{), }
      \AttributeTok{labels =} \FunctionTok{c}\NormalTok{(}\StringTok{"\textless{}=1"}\NormalTok{,}\StringTok{"2{-}3"}\NormalTok{,}\StringTok{"4{-}5"}\NormalTok{,}\StringTok{"6{-}10"}\NormalTok{,}\StringTok{"11{-}24"}\NormalTok{,}\StringTok{"25\textgreater{}="}\NormalTok{))}
\NormalTok{data\_set[}\StringTok{"SEG\_Precio\_M2\_total"}\NormalTok{] }\OtherTok{\textless{}{-}} 
  \FunctionTok{cut}\NormalTok{(data\_set}\SpecialCharTok{$}\NormalTok{Precio\_M2\_total, }\AttributeTok{breaks =} \FunctionTok{c}\NormalTok{(}\DecValTok{0}\NormalTok{,}\DecValTok{1}\NormalTok{,}\DecValTok{500}\NormalTok{,}\DecValTok{1000}\NormalTok{,}\DecValTok{2000}\NormalTok{,}\DecValTok{5000}\NormalTok{,}\DecValTok{8000}\NormalTok{,}\DecValTok{10000}\NormalTok{), }
      \AttributeTok{labels =} \FunctionTok{c}\NormalTok{(}\StringTok{"\textless{}=1"}\NormalTok{,}\StringTok{"2{-}500"}\NormalTok{,}\StringTok{"501{-}1000"}\NormalTok{,}\StringTok{"1001{-}2000"}\NormalTok{,}\StringTok{"2001{-}5000"}\NormalTok{,}\StringTok{"5001{-}8000"}\NormalTok{,}\StringTok{"8001\textgreater{}="}\NormalTok{))}
\NormalTok{data\_set[}\StringTok{"SEG\_Precio"}\NormalTok{] }\OtherTok{\textless{}{-}} 
  \FunctionTok{cut}\NormalTok{(data\_set}\SpecialCharTok{$}\NormalTok{PRICE, }\AttributeTok{breaks =} \FunctionTok{c}\NormalTok{(}\DecValTok{0}\NormalTok{,}\DecValTok{500}\NormalTok{,}\DecValTok{750}\NormalTok{,}\DecValTok{1000}\NormalTok{,}\DecValTok{1500}\NormalTok{,}\DecValTok{2000}\NormalTok{,}\DecValTok{3000}\NormalTok{), }
      \AttributeTok{labels =} \FunctionTok{c}\NormalTok{(}\StringTok{"\textless{}=500"}\NormalTok{,}\StringTok{"501{-}750"}\NormalTok{,}\StringTok{"751{-}1000"}\NormalTok{,}\StringTok{"1001{-}1500"}\NormalTok{,}\StringTok{"1501{-}2000"}\NormalTok{,}\StringTok{"2001\textgreater{}="}\NormalTok{))}

\NormalTok{data\_set}\SpecialCharTok{$}\NormalTok{SEG\_BUILD\_YEAR }\OtherTok{\textless{}{-}} \FunctionTok{as.factor}\NormalTok{(data\_set}\SpecialCharTok{$}\NormalTok{SEG\_BUILD\_YEAR)}
\NormalTok{data\_set}\SpecialCharTok{$}\NormalTok{SEG\_CBD\_DIST }\OtherTok{\textless{}{-}} \FunctionTok{as.factor}\NormalTok{(data\_set}\SpecialCharTok{$}\NormalTok{SEG\_CBD\_DIST)}
\NormalTok{data\_set}\SpecialCharTok{$}\NormalTok{SEG\_NEAREST\_STN\_DIST }\OtherTok{\textless{}{-}} \FunctionTok{as.factor}\NormalTok{(data\_set}\SpecialCharTok{$}\NormalTok{SEG\_NEAREST\_STN\_DIST)}
\NormalTok{data\_set}\SpecialCharTok{$}\NormalTok{SEG\_NEAREST\_SCH\_DIST }\OtherTok{\textless{}{-}} \FunctionTok{as.factor}\NormalTok{(data\_set}\SpecialCharTok{$}\NormalTok{SEG\_NEAREST\_SCH\_DIST)}
\NormalTok{data\_set}\SpecialCharTok{$}\NormalTok{SEG\_Precio\_M2\_total }\OtherTok{\textless{}{-}} \FunctionTok{as.factor}\NormalTok{(data\_set}\SpecialCharTok{$}\NormalTok{SEG\_Precio\_M2\_total)}
\NormalTok{data\_set}\SpecialCharTok{$}\NormalTok{SEG\_Precio }\OtherTok{\textless{}{-}} \FunctionTok{as.factor}\NormalTok{(data\_set}\SpecialCharTok{$}\NormalTok{SEG\_Precio)}
\end{Highlighting}
\end{Shaded}

\hypertarget{selecciuxf3n-de-los-grupos-de-datos-que-se-quieren-analizarcomparar}{%
\subsubsection{Selección de los grupos de datos que se quieren
analizar/comparar}\label{selecciuxf3n-de-los-grupos-de-datos-que-se-quieren-analizarcomparar}}

Para un conocimiento mayor sobre los datos, tenemos a nuestro alcance
unas herramientas muy valiosas: las herramientas de visualización. Para
dichas visualizaciones, haremos uso de los paquetes ggplot2, gridExtra y
grid de R.

Siempre es importante analizar los datos que tenemos ya que las
conclusiones dependerán de las características de la muestra.

\begin{Shaded}
\begin{Highlighting}[]
\FunctionTok{grid.newpage}\NormalTok{()}
\NormalTok{plotbyChecking\_balance}\OtherTok{\textless{}{-}}\FunctionTok{ggplot}\NormalTok{(data\_set,}\FunctionTok{aes}\NormalTok{(SEG\_BUILD\_YEAR ))}\SpecialCharTok{+}\FunctionTok{geom\_bar}\NormalTok{() }\SpecialCharTok{+}\FunctionTok{labs}\NormalTok{(}\AttributeTok{x=}\StringTok{"$SEG\_BUILD\_YEAR"}\NormalTok{, }\AttributeTok{y=}\StringTok{"Viviendas"}\NormalTok{)}\SpecialCharTok{+} \FunctionTok{guides}\NormalTok{(}\AttributeTok{fill=}\FunctionTok{guide\_legend}\NormalTok{(}\AttributeTok{title=}\StringTok{""}\NormalTok{))}\SpecialCharTok{+} \FunctionTok{scale\_fill\_manual}\NormalTok{(}\AttributeTok{values=}\FunctionTok{c}\NormalTok{(}\StringTok{"blue"}\NormalTok{,}\StringTok{"\#008000"}\NormalTok{))}\SpecialCharTok{+}\FunctionTok{ggtitle}\NormalTok{(}\StringTok{"$SEG\_BUILD\_YEAR "}\NormalTok{)}

\NormalTok{plotbyCredit\_history}\OtherTok{\textless{}{-}}\FunctionTok{ggplot}\NormalTok{(data\_set,}\FunctionTok{aes}\NormalTok{(SEG\_CBD\_DIST))}\SpecialCharTok{+}\FunctionTok{geom\_bar}\NormalTok{() }\SpecialCharTok{+}\FunctionTok{labs}\NormalTok{(}\AttributeTok{x=}\StringTok{"SEG\_CBD\_DIST"}\NormalTok{, }\AttributeTok{y=}\StringTok{"Viviendas"}\NormalTok{)}\SpecialCharTok{+} \FunctionTok{guides}\NormalTok{(}\AttributeTok{fill=}\FunctionTok{guide\_legend}\NormalTok{(}\AttributeTok{title=}\StringTok{""}\NormalTok{))}\SpecialCharTok{+} \FunctionTok{scale\_fill\_manual}\NormalTok{(}\AttributeTok{values=}\FunctionTok{c}\NormalTok{(}\StringTok{"blue"}\NormalTok{,}\StringTok{"\#008000"}\NormalTok{))}\SpecialCharTok{+}\FunctionTok{ggtitle}\NormalTok{(}\StringTok{"SEG\_CBD\_DIST"}\NormalTok{)}

\NormalTok{plotbysavings\_balance}\OtherTok{\textless{}{-}}\FunctionTok{ggplot}\NormalTok{(data\_set,}\FunctionTok{aes}\NormalTok{(SEG\_NEAREST\_STN\_DIST))}\SpecialCharTok{+}\FunctionTok{geom\_bar}\NormalTok{() }\SpecialCharTok{+}\FunctionTok{labs}\NormalTok{(}\AttributeTok{x=}\StringTok{"SEG\_NEAREST\_STN\_DIST"}\NormalTok{, }\AttributeTok{y=}\StringTok{"Viviendas"}\NormalTok{)}\SpecialCharTok{+} \FunctionTok{guides}\NormalTok{(}\AttributeTok{fill=}\FunctionTok{guide\_legend}\NormalTok{(}\AttributeTok{title=}\StringTok{""}\NormalTok{))}\SpecialCharTok{+} \FunctionTok{scale\_fill\_manual}\NormalTok{(}\AttributeTok{values=}\FunctionTok{c}\NormalTok{(}\StringTok{"blue"}\NormalTok{,}\StringTok{"\#008000"}\NormalTok{))}\SpecialCharTok{+}\FunctionTok{ggtitle}\NormalTok{(}\StringTok{"SEG\_NEAREST\_STN\_DIST"}\NormalTok{)}

\NormalTok{plotbyemployment\_length}\OtherTok{\textless{}{-}}\FunctionTok{ggplot}\NormalTok{(data\_set,}\FunctionTok{aes}\NormalTok{(SEG\_NEAREST\_SCH\_DIST))}\SpecialCharTok{+}\FunctionTok{geom\_bar}\NormalTok{() }\SpecialCharTok{+}\FunctionTok{labs}\NormalTok{(}\AttributeTok{x=}\StringTok{"SEG\_NEAREST\_SCH\_DIST"}\NormalTok{, }\AttributeTok{y=}\StringTok{"Viviendas"}\NormalTok{)}\SpecialCharTok{+} \FunctionTok{guides}\NormalTok{(}\AttributeTok{fill=}\FunctionTok{guide\_legend}\NormalTok{(}\AttributeTok{title=}\StringTok{""}\NormalTok{))}\SpecialCharTok{+} \FunctionTok{scale\_fill\_manual}\NormalTok{(}\AttributeTok{values=}\FunctionTok{c}\NormalTok{(}\StringTok{"blue"}\NormalTok{,}\StringTok{"\#008000"}\NormalTok{))}\SpecialCharTok{+}\FunctionTok{ggtitle}\NormalTok{(}\StringTok{"SEG\_NEAREST\_SCH\_DIST"}\NormalTok{)}

\FunctionTok{grid.arrange}\NormalTok{(plotbysavings\_balance,plotbyCredit\_history,plotbyemployment\_length,plotbyChecking\_balance,}\AttributeTok{ncol=}\DecValTok{2}\NormalTok{)}
\end{Highlighting}
\end{Shaded}

\includegraphics{sguaigua_PRA2_files/figure-latex/unnamed-chunk-26-1.pdf}

\begin{Shaded}
\begin{Highlighting}[]
\FunctionTok{grid.newpage}\NormalTok{()}
\NormalTok{plotbypurpose}\OtherTok{\textless{}{-}}\FunctionTok{ggplot}\NormalTok{(data\_set,}\FunctionTok{aes}\NormalTok{(SEG\_Precio\_M2\_total ))}\SpecialCharTok{+}\FunctionTok{geom\_bar}\NormalTok{() }\SpecialCharTok{+}\FunctionTok{labs}\NormalTok{(}\AttributeTok{x=}\StringTok{"SEG\_Precio\_M2\_total"}\NormalTok{, }\AttributeTok{y=}\StringTok{"Viviendas"}\NormalTok{)}\SpecialCharTok{+} \FunctionTok{guides}\NormalTok{(}\AttributeTok{fill=}\FunctionTok{guide\_legend}\NormalTok{(}\AttributeTok{title=}\StringTok{""}\NormalTok{))}\SpecialCharTok{+} \FunctionTok{scale\_fill\_manual}\NormalTok{(}\AttributeTok{values=}\FunctionTok{c}\NormalTok{(}\StringTok{"blue"}\NormalTok{,}\StringTok{"\#008000"}\NormalTok{))}\SpecialCharTok{+}\FunctionTok{ggtitle}\NormalTok{(}\StringTok{"SEG\_Precio\_M2\_total"}\NormalTok{)}

\NormalTok{plotbyother\_debtors}\OtherTok{\textless{}{-}}\FunctionTok{ggplot}\NormalTok{(data\_set,}\FunctionTok{aes}\NormalTok{(SUBURB))}\SpecialCharTok{+}\FunctionTok{geom\_bar}\NormalTok{() }\SpecialCharTok{+}\FunctionTok{labs}\NormalTok{(}\AttributeTok{x=}\StringTok{"SUBURB"}\NormalTok{, }\AttributeTok{y=}\StringTok{"Viviendas"}\NormalTok{)}\SpecialCharTok{+} \FunctionTok{guides}\NormalTok{(}\AttributeTok{fill=}\FunctionTok{guide\_legend}\NormalTok{(}\AttributeTok{title=}\StringTok{""}\NormalTok{))}\SpecialCharTok{+} \FunctionTok{scale\_fill\_manual}\NormalTok{(}\AttributeTok{values=}\FunctionTok{c}\NormalTok{(}\StringTok{"blue"}\NormalTok{,}\StringTok{"\#008000"}\NormalTok{))}\SpecialCharTok{+}\FunctionTok{ggtitle}\NormalTok{(}\StringTok{"SUBURB"}\NormalTok{)}

\FunctionTok{grid.arrange}\NormalTok{(plotbypurpose,plotbyother\_debtors,}\AttributeTok{ncol=}\DecValTok{2}\NormalTok{)}
\end{Highlighting}
\end{Shaded}

\includegraphics{sguaigua_PRA2_files/figure-latex/unnamed-chunk-27-1.pdf}

La varialbe SUBURB al ser una variable del tipo Char no se ha podido
categorizar, muestra una grafica no entendible. Eliminamos la variable
SUBURB.

\begin{Shaded}
\begin{Highlighting}[]
\NormalTok{data\_set[}\FunctionTok{c}\NormalTok{(}\StringTok{"SUBURB"}\NormalTok{)] }\OtherTok{\textless{}{-}} \ConstantTok{NULL}
\end{Highlighting}
\end{Shaded}

La variable NEAREST\_SCH\_DIST, se identifica que hay mas viviendas
compradas cuando una escula se encuentra mas cerca.

\hypertarget{comprobaciuxf3n-de-la-normalidad-y-homogeneidad-de-la-varianza.}{%
\subsubsection{Comprobación de la normalidad y homogeneidad de la
varianza.}\label{comprobaciuxf3n-de-la-normalidad-y-homogeneidad-de-la-varianza.}}

Para la comprobación de que los valores que toman nuestras variables
cuantitativas provienen de una población distribuida normalmente,
utilizaremos la prueba de normalidad de AndersonDarling. Así, se
comprueba que para que cada prueba se obtiene un p-valor superior al
nivel de significación prefijado α = 0, 05. Si esto se cumple, entonces
se considera que variable en cuestión sigue una distribución normal. Se
identifica que ninguna variable es normal.

\begin{Shaded}
\begin{Highlighting}[]
\NormalTok{alpha }\OtherTok{=} \FloatTok{0.05}
\NormalTok{col.names }\OtherTok{=} \FunctionTok{colnames}\NormalTok{(data\_set)}
\ControlFlowTok{for}\NormalTok{ (i }\ControlFlowTok{in} \DecValTok{1}\SpecialCharTok{:}\FunctionTok{ncol}\NormalTok{(data\_set)) \{}
  \ControlFlowTok{if}\NormalTok{ (i }\SpecialCharTok{==} \DecValTok{1}\NormalTok{) }\FunctionTok{cat}\NormalTok{(}\StringTok{"Variables que no siguen una distribución normal:}\SpecialCharTok{\textbackslash{}n}\StringTok{"}\NormalTok{)}
  \ControlFlowTok{if}\NormalTok{ (}\FunctionTok{is.integer}\NormalTok{(data\_set[,i]) }\SpecialCharTok{|} \FunctionTok{is.numeric}\NormalTok{(data\_set[,i])) \{}
\NormalTok{    p\_val }\OtherTok{=} \FunctionTok{ad.test}\NormalTok{(data\_set[,i])}\SpecialCharTok{$}\NormalTok{p.value}
    \ControlFlowTok{if}\NormalTok{ (p\_val }\SpecialCharTok{\textless{}}\NormalTok{ alpha) \{}
      \FunctionTok{cat}\NormalTok{(col.names[i])}
      \CommentTok{\# Format output}
      \ControlFlowTok{if}\NormalTok{ (i }\SpecialCharTok{\textless{}} \FunctionTok{ncol}\NormalTok{(data\_set) }\SpecialCharTok{{-}} \DecValTok{1}\NormalTok{) }\FunctionTok{cat}\NormalTok{(}\StringTok{", "}\NormalTok{)}
      \ControlFlowTok{if}\NormalTok{ (i }\SpecialCharTok{\%\%} \DecValTok{3} \SpecialCharTok{==} \DecValTok{0}\NormalTok{) }\FunctionTok{cat}\NormalTok{(}\StringTok{"}\SpecialCharTok{\textbackslash{}n}\StringTok{"}\NormalTok{)}
\NormalTok{    \}}
\NormalTok{  \}}
\NormalTok{\}}
\end{Highlighting}
\end{Shaded}

\begin{verbatim}
## Variables que no siguen una distribución normal:
## PRICE, BUILD_YEAR, CBD_DIST, 
## NEAREST_STN_DIST, NEAREST_SCH_DIST, Precio_M2_total,
\end{verbatim}

Normalizamos utilizando la función ``rescale'' que realiza un escalado
de la variables en un rango {[}0,1{]} es decir, por la diferencia:

\begin{Shaded}
\begin{Highlighting}[]
\NormalTok{data\_set\_norm }\OtherTok{\textless{}{-}}\NormalTok{ data\_set[}\FunctionTok{c}\NormalTok{( }\StringTok{"CBD\_DIST"}\NormalTok{,}\StringTok{"NEAREST\_STN\_DIST"}\NormalTok{,}\StringTok{"NEAREST\_SCH\_DIST"}\NormalTok{,}\StringTok{"PRICE"}\NormalTok{,}\StringTok{"BUILD\_YEAR"}\NormalTok{)]}

\NormalTok{data\_set\_norm }\OtherTok{\textless{}{-}} \FunctionTok{sapply}\NormalTok{(data\_set\_norm, rescale)}
\FunctionTok{head}\NormalTok{(data\_set\_norm)}
\end{Highlighting}
\end{Shaded}

\begin{verbatim}
##       CBD_DIST NEAREST_STN_DIST NEAREST_SCH_DIST     PRICE BUILD_YEAR
## [1,] 0.3600394       0.16778267        0.1967326 0.4165316  0.8055556
## [2,] 0.4482391       0.17734838        0.4099359 0.1912480  0.4722222
## [3,] 0.3518348       0.33996556        0.3897342 0.1653160  0.1111111
## [4,] 0.2144073       0.18691410        0.3750644 0.2220421  0.7361111
## [5,] 0.5446434       0.09125694        0.3003371 0.2901135  0.6388889
## [6,] 0.5631038       0.34953128        0.6272206 0.2828201  0.9583333
\end{verbatim}

Seguidamente, pasamos a estudiar la homogeneidad de varianzas mediante
la aplicación de un test de Fligner-Killeen. En este caso, estudiaremos
esta homogeneidad en cuanto a los grupos conformados por los vehículos
que presentan un motor turbo frente a un motor estándar. En el siguiente
test, la hipótesis nula consiste en que ambas varianzas son iguales.

\begin{Shaded}
\begin{Highlighting}[]
\FunctionTok{fligner.test}\NormalTok{(PRICE }\SpecialCharTok{\textasciitilde{}}\NormalTok{ NEAREST\_STN\_DIST, }\AttributeTok{data =}\NormalTok{ data\_set\_norm)}
\end{Highlighting}
\end{Shaded}

\begin{verbatim}
## 
##  Fligner-Killeen test of homogeneity of variances
## 
## data:  PRICE by NEAREST_STN_DIST
## Fligner-Killeen:med chi-squared = 1908.9, df = 900, p-value < 2.2e-16
\end{verbatim}

\begin{Shaded}
\begin{Highlighting}[]
\FunctionTok{fligner.test}\NormalTok{(PRICE }\SpecialCharTok{\textasciitilde{}}\NormalTok{ NEAREST\_SCH\_DIST, }\AttributeTok{data =}\NormalTok{ data\_set\_norm)}
\end{Highlighting}
\end{Shaded}

\begin{verbatim}
## 
##  Fligner-Killeen test of homogeneity of variances
## 
## data:  PRICE by NEAREST_SCH_DIST
## Fligner-Killeen:med chi-squared = 24173, df = 24997, p-value = 0.9999
\end{verbatim}

\begin{Shaded}
\begin{Highlighting}[]
\FunctionTok{fligner.test}\NormalTok{(PRICE }\SpecialCharTok{\textasciitilde{}}\NormalTok{ CBD\_DIST, }\AttributeTok{data =}\NormalTok{ data\_set\_norm)}
\end{Highlighting}
\end{Shaded}

\begin{verbatim}
## 
##  Fligner-Killeen test of homogeneity of variances
## 
## data:  PRICE by CBD_DIST
## Fligner-Killeen:med chi-squared = 1982.3, df = 491, p-value < 2.2e-16
\end{verbatim}

\begin{Shaded}
\begin{Highlighting}[]
\FunctionTok{fligner.test}\NormalTok{(PRICE }\SpecialCharTok{\textasciitilde{}}\NormalTok{ BUILD\_YEAR, }\AttributeTok{data =}\NormalTok{ data\_set\_norm)}
\end{Highlighting}
\end{Shaded}

\begin{verbatim}
## 
##  Fligner-Killeen test of homogeneity of variances
## 
## data:  PRICE by BUILD_YEAR
## Fligner-Killeen:med chi-squared = 463.54, df = 72, p-value < 2.2e-16
\end{verbatim}

Solo para la variable NEAREST\_SCH\_DIST, obtenemos un p-valor superior
a 0,05, aceptamos la hipótesis de que las varianzas de ambas muestras
son homogéneas.

\hypertarget{aplicaciuxf3n-de-pruebas-estaduxedsticas-para-comparar-los-grupos-de-datos.}{%
\subsubsection{Aplicación de pruebas estadísticas para comparar los
grupos de
datos.}\label{aplicaciuxf3n-de-pruebas-estaduxedsticas-para-comparar-los-grupos-de-datos.}}

\textbf{Analisis de Correlaciones}

Procedemos a realizar un análisis de correlación entre las distintas
variables para determinar cuáles de ellas ejercen una mayor influencia
sobre el precio final de la vivienda. Para ello, se utilizará el
coeficiente de correlación de Spearman, puesto que hemos visto que
tenemos datos que no siguen una distribución normal. Se identifica que
el precio se relaciona con la varible NEAREST\_SCH\_DIST

\begin{Shaded}
\begin{Highlighting}[]
\FunctionTok{cor.test}\NormalTok{(data\_set}\SpecialCharTok{$}\NormalTok{PRICE,data\_set}\SpecialCharTok{$}\NormalTok{NEAREST\_STN\_DIST, }\AttributeTok{method=}\StringTok{"spearman"}\NormalTok{)}
\end{Highlighting}
\end{Shaded}

\begin{verbatim}
## Warning in cor.test.default(data_set$PRICE, data_set$NEAREST_STN_DIST, method =
## "spearman"): Cannot compute exact p-value with ties
\end{verbatim}

\begin{verbatim}
## 
##  Spearman's rank correlation rho
## 
## data:  data_set$PRICE and data_set$NEAREST_STN_DIST
## S = 2.5712e+12, p-value = 6.722e-12
## alternative hypothesis: true rho is not equal to 0
## sample estimates:
##        rho 
## 0.04316365
\end{verbatim}

\begin{Shaded}
\begin{Highlighting}[]
\FunctionTok{cor.test}\NormalTok{(data\_set}\SpecialCharTok{$}\NormalTok{PRICE,data\_set}\SpecialCharTok{$}\NormalTok{CBD\_DIST, }\AttributeTok{method=}\StringTok{"spearman"}\NormalTok{)}
\end{Highlighting}
\end{Shaded}

\begin{verbatim}
## Warning in cor.test.default(data_set$PRICE, data_set$CBD_DIST, method = "spearman"):
## Cannot compute exact p-value with ties
\end{verbatim}

\begin{verbatim}
## 
##  Spearman's rank correlation rho
## 
## data:  data_set$PRICE and data_set$CBD_DIST
## S = 3.5781e+12, p-value < 2.2e-16
## alternative hypothesis: true rho is not equal to 0
## sample estimates:
##        rho 
## -0.3315394
\end{verbatim}

\begin{Shaded}
\begin{Highlighting}[]
\FunctionTok{cor.test}\NormalTok{(data\_set}\SpecialCharTok{$}\NormalTok{PRICE,data\_set}\SpecialCharTok{$}\NormalTok{NEAREST\_SCH\_DIST, }\AttributeTok{method=}\StringTok{"spearman"}\NormalTok{)}
\end{Highlighting}
\end{Shaded}

\begin{verbatim}
## Warning in cor.test.default(data_set$PRICE, data_set$NEAREST_SCH_DIST, method =
## "spearman"): Cannot compute exact p-value with ties
\end{verbatim}

\begin{verbatim}
## 
##  Spearman's rank correlation rho
## 
## data:  data_set$PRICE and data_set$NEAREST_SCH_DIST
## S = 2.5976e+12, p-value = 1.142e-07
## alternative hypothesis: true rho is not equal to 0
## sample estimates:
##       rho 
## 0.0333534
\end{verbatim}

\begin{Shaded}
\begin{Highlighting}[]
\FunctionTok{cor.test}\NormalTok{(data\_set}\SpecialCharTok{$}\NormalTok{Precio\_M2\_total,data\_set}\SpecialCharTok{$}\NormalTok{NEAREST\_SCH\_DIST, }\AttributeTok{method=}\StringTok{"spearman"}\NormalTok{)}
\end{Highlighting}
\end{Shaded}

\begin{verbatim}
## Warning in cor.test.default(data_set$Precio_M2_total, data_set$NEAREST_SCH_DIST, : Cannot
## compute exact p-value with ties
\end{verbatim}

\begin{verbatim}
## 
##  Spearman's rank correlation rho
## 
## data:  data_set$Precio_M2_total and data_set$NEAREST_SCH_DIST
## S = 2.7978e+12, p-value = 5.968e-11
## alternative hypothesis: true rho is not equal to 0
## sample estimates:
##        rho 
## -0.0411598
\end{verbatim}

\begin{Shaded}
\begin{Highlighting}[]
\FunctionTok{cor.test}\NormalTok{(data\_set}\SpecialCharTok{$}\NormalTok{PRICE,data\_set}\SpecialCharTok{$}\NormalTok{BUILD\_YEAR, }\AttributeTok{method=}\StringTok{"spearman"}\NormalTok{)}
\end{Highlighting}
\end{Shaded}

\begin{verbatim}
## Warning in cor.test.default(data_set$PRICE, data_set$BUILD_YEAR, method = "spearman"):
## Cannot compute exact p-value with ties
\end{verbatim}

\begin{verbatim}
## 
##  Spearman's rank correlation rho
## 
## data:  data_set$PRICE and data_set$BUILD_YEAR
## S = 2.701e+12, p-value = 0.4148
## alternative hypothesis: true rho is not equal to 0
## sample estimates:
##          rho 
## -0.005130821
\end{verbatim}

Se mantiene la relación entre el precio y la cercania a una escuela
NEAREST\_SCH\_DIST y CBD\_DIST.

\begin{Shaded}
\begin{Highlighting}[]
\NormalTok{  plot\_dot\_plot\_precio }\OtherTok{\textless{}{-}} \ControlFlowTok{function}\NormalTok{(data\_set,color )\{}
\NormalTok{  p3}\OtherTok{\textless{}{-}}\FunctionTok{ggplot}\NormalTok{(}\AttributeTok{data =}\NormalTok{ data\_set, }\FunctionTok{aes\_string}\NormalTok{(}\AttributeTok{x=}\StringTok{"NEAREST\_SCH\_DIST"}\NormalTok{, }\AttributeTok{y=}\StringTok{"CBD\_DIST"}\NormalTok{)) }\SpecialCharTok{+}
             \FunctionTok{geom\_point}\NormalTok{(}\FunctionTok{aes\_string}\NormalTok{(}\AttributeTok{colour=}\NormalTok{color))}
\NormalTok{  lay }\OtherTok{\textless{}{-}} \FunctionTok{rbind}\NormalTok{(}\FunctionTok{c}\NormalTok{(}\DecValTok{1}\NormalTok{,}\DecValTok{2}\NormalTok{))}
  \FunctionTok{grid.arrange}\NormalTok{(p3, }\AttributeTok{nrow =} \DecValTok{1}\NormalTok{,}\AttributeTok{layout\_matrix =}\NormalTok{ lay)}
\NormalTok{\}}

\FunctionTok{plot\_dot\_plot\_precio}\NormalTok{(}\FunctionTok{na.omit}\NormalTok{(data\_set),}\StringTok{"SEG\_Precio"}\NormalTok{)}
\end{Highlighting}
\end{Shaded}

\includegraphics{sguaigua_PRA2_files/figure-latex/unnamed-chunk-33-1.pdf}

\textbf{Regresión Lineal}

Revisión de relación entre variables se reconfirma que las variables
``CBD\_DIST'',``NEAREST\_STN\_DIST'',``NEAREST\_SCH\_DIST'' son las que
mas se relacionan con el precio.

Se calculará un modelo de regresión lineal utilizando regresores
cuantitativos para poder realizar las predicciones de los precios.

Regresores cuantitativos con mayor coeficiente de correlación con
respecto al precio:
``CBD\_DIST'',``NEAREST\_STN\_DIST'',``NEAREST\_SCH\_DIST''

\begin{Shaded}
\begin{Highlighting}[]
\NormalTok{distancia\_centro}\OtherTok{=}\NormalTok{data\_set}\SpecialCharTok{$}\NormalTok{CBD\_DIST}
\NormalTok{distancia\_estacion\_bus}\OtherTok{=}\NormalTok{data\_set}\SpecialCharTok{$}\NormalTok{NEAREST\_STN\_DIST}
\NormalTok{distancia\_escuela}\OtherTok{=}\NormalTok{data\_set}\SpecialCharTok{$}\NormalTok{NEAREST\_SCH\_DIST}
\end{Highlighting}
\end{Shaded}

Variable a predecir ``PRECIO'' por metro cuadrado

\begin{Shaded}
\begin{Highlighting}[]
\NormalTok{precio}\OtherTok{=}\NormalTok{data\_set}\SpecialCharTok{$}\NormalTok{Precio\_M2\_total}
\end{Highlighting}
\end{Shaded}

\begin{Shaded}
\begin{Highlighting}[]
\NormalTok{modelo1 }\OtherTok{=} \FunctionTok{lm}\NormalTok{(precio}\SpecialCharTok{\textasciitilde{}}\NormalTok{distancia\_escuela,}\AttributeTok{data=}\NormalTok{data\_set)}
\NormalTok{modelo2 }\OtherTok{=} \FunctionTok{lm}\NormalTok{(precio}\SpecialCharTok{\textasciitilde{}}\NormalTok{distancia\_escuela}\SpecialCharTok{+}\NormalTok{distancia\_centro,}\AttributeTok{data=}\NormalTok{data\_set)}
\NormalTok{modelo3 }\OtherTok{=} \FunctionTok{lm}\NormalTok{(precio}\SpecialCharTok{\textasciitilde{}}\NormalTok{distancia\_escuela}\SpecialCharTok{+}\NormalTok{distancia\_estacion\_bus,}\AttributeTok{data=}\NormalTok{data\_set)}
\NormalTok{modelo4 }\OtherTok{=} \FunctionTok{lm}\NormalTok{(precio}\SpecialCharTok{\textasciitilde{}}\NormalTok{distancia\_escuela}\SpecialCharTok{+}\NormalTok{distancia\_estacion\_bus}\SpecialCharTok{+}\NormalTok{distancia\_centro,}\AttributeTok{data=}\NormalTok{data\_set)}
\NormalTok{modelo5 }\OtherTok{=} \FunctionTok{lm}\NormalTok{(precio}\SpecialCharTok{\textasciitilde{}}\NormalTok{distancia\_estacion\_bus}\SpecialCharTok{+}\NormalTok{distancia\_centro,}\AttributeTok{data=}\NormalTok{data\_set)}
\NormalTok{modelo6 }\OtherTok{=} \FunctionTok{lm}\NormalTok{(precio}\SpecialCharTok{\textasciitilde{}}\NormalTok{distancia\_centro,}\AttributeTok{data=}\NormalTok{data\_set)}
\NormalTok{modelo7 }\OtherTok{=} \FunctionTok{lm}\NormalTok{(precio}\SpecialCharTok{\textasciitilde{}}\NormalTok{distancia\_estacion\_bus,}\AttributeTok{data=}\NormalTok{data\_set)}

\CommentTok{\# Tabla con los coeficientes de determinación de cada modelo}
\NormalTok{tabla.coeficientes }\OtherTok{\textless{}{-}} \FunctionTok{matrix}\NormalTok{(}\FunctionTok{c}\NormalTok{(}\DecValTok{1}\NormalTok{, }\FunctionTok{summary}\NormalTok{(modelo1)}\SpecialCharTok{$}\NormalTok{r.squared,}
\DecValTok{2}\NormalTok{, }\FunctionTok{summary}\NormalTok{(modelo2)}\SpecialCharTok{$}\NormalTok{r.squared,}
\DecValTok{3}\NormalTok{, }\FunctionTok{summary}\NormalTok{(modelo3)}\SpecialCharTok{$}\NormalTok{r.squared,}
\DecValTok{4}\NormalTok{, }\FunctionTok{summary}\NormalTok{(modelo4)}\SpecialCharTok{$}\NormalTok{r.squared,}
\DecValTok{5}\NormalTok{, }\FunctionTok{summary}\NormalTok{(modelo5)}\SpecialCharTok{$}\NormalTok{r.squared,}
\DecValTok{6}\NormalTok{, }\FunctionTok{summary}\NormalTok{(modelo6)}\SpecialCharTok{$}\NormalTok{r.squared,}
\DecValTok{7}\NormalTok{, }\FunctionTok{summary}\NormalTok{(modelo7)}\SpecialCharTok{$}\NormalTok{r.squared),}
\AttributeTok{ncol =} \DecValTok{2}\NormalTok{, }\AttributeTok{byrow =} \ConstantTok{TRUE}\NormalTok{)}
\FunctionTok{colnames}\NormalTok{(tabla.coeficientes) }\OtherTok{\textless{}{-}} \FunctionTok{c}\NormalTok{(}\StringTok{"Modelo"}\NormalTok{, }\StringTok{"R\^{}2"}\NormalTok{)}
\NormalTok{tabla.coeficientes}
\end{Highlighting}
\end{Shaded}

\begin{verbatim}
##      Modelo         R^2
## [1,]      1 0.004489146
## [2,]      2 0.108002288
## [3,]      3 0.030230581
## [4,]      4 0.134728270
## [5,]      5 0.134474083
## [6,]      6 0.106998037
## [7,]      7 0.029882086
\end{verbatim}

En este caso, tenemos que el cuarto modelo es el más conveniente dado
que tiene un mayor coeficiente de determinación. Sin embargo el valor
del coeficiente es bastante bajo.

Ahora, empleando este modelo, podemos proceder a realizar predicciones
de precios de vehículos como el siguiente:

\begin{Shaded}
\begin{Highlighting}[]
\NormalTok{newdata }\OtherTok{\textless{}{-}} \FunctionTok{data.frame}\NormalTok{(}
\AttributeTok{distancia\_centro =} \FloatTok{17.50}\NormalTok{,}
\AttributeTok{distancia\_estacion\_bus =} \FloatTok{3.2}\NormalTok{,}
\AttributeTok{distancia\_escuela =} \FloatTok{1.35}
\NormalTok{)}
\CommentTok{\# Predecir el precio}
\FunctionTok{predict}\NormalTok{(modelo4, newdata)}
\end{Highlighting}
\end{Shaded}

\begin{verbatim}
##        1 
## 904.7931
\end{verbatim}

En función de los datos y el objetivo del estudio, aplicar pruebas de
contraste de hipótesis, correlaciones, regresiones, etc. Aplicar al
menos tres métodos de análisis diferentes.

\hypertarget{representaciuxf3n-de-los-resultados-a-partir-de-tablas-y-gruxe1ficas.}{%
\subsection{Representación de los resultados a partir de tablas y
gráficas.}\label{representaciuxf3n-de-los-resultados-a-partir-de-tablas-y-gruxe1ficas.}}

Este apartado se puede responder a lo largo de la práctica, sin
necesidad de concentrar todas las representaciones en este punto de la
práctica.

\hypertarget{cuuxe1les-son-las-conclusiones}{%
\subsubsection{¿cuáles son las
conclusiones?}\label{cuuxe1les-son-las-conclusiones}}

Inicialmente se han realizado tareas de limpieza, identificando campos
vacíos y outliers, se eliminaron variables que no tenían suficiente
relación con las variables. Se procede a normalizar los valores pues
estaban con valores muy altos. Mediante la discretización,se puede
observar las variables mas importantes que interfieren para identificar
el precio de una viviendo, llegando a la conclusión que las variables
mas importantes son cercanía a la escuela SEG\_NEAREST\_SCH\_DIST y
cercania al centro(CBD\_DIST). En base a estas variables se pudo
predecir el valor de una construcción.

Los modelos son faciles de implementar con los algoritmos que presta los
paquetes, sin embargo si se debe tener mucho cuidado con las variables
del set de datos que se va ha manejar, y esto depende de un buen
analisis de la información y comparativa entre variables, adicionalmente
del tratamiento previo que se debe dar a los mismos.

\hypertarget{los-resultados-permiten-responder-al-problema}{%
\subsubsection{¿Los resultados permiten responder al
problema?}\label{los-resultados-permiten-responder-al-problema}}

Si se pudo predecir el valor de una variable, en base a las variables
que mas influyen en el calculo.

\hypertarget{cuxf3digo-hay-que-adjuntar-el-cuxf3digo-preferiblemente-en-r}{%
\subsection{Código: Hay que adjuntar el código, preferiblemente en
R,}\label{cuxf3digo-hay-que-adjuntar-el-cuxf3digo-preferiblemente-en-r}}

\begin{Shaded}
\begin{Highlighting}[]
\FunctionTok{write.csv}\NormalTok{(data\_set, }\AttributeTok{file=}\StringTok{"data\_out.csv"}\NormalTok{)}
\end{Highlighting}
\end{Shaded}


\end{document}
